\documentclass[a4 paper, 12pt]{article}
\pagenumbering{gobble}

\begin{document}
\begin{center}
\Large{\textbf{PROJECT ABSTRACT}}
\\
\textbf{Title: Bus Route Navigator}
\end{center}
\begin{flushleft}
The most preferred mode of transportation for the citizens is the local bus service. The reason behind their wide usage is, the cheap ticket fares, the convenience as well as the safety being ensured for passengers. Being a local service, the buses display the destinations in their native language. Unfortunately, foreigners along with people who are unable to comprehend the native language find it difficult to use the local bus service.
\\
\vspace{6pt}
The project focuses on developing a software solution that will help end users to navigate across a city by integrating with the local bus service. Image of the bus board is captured via a phone camera. They are preprocessed using various image enhancement operations in order to obtain the image in the required format. Optical Character Recognition which is implemented by using Convolutional Neural Network, is used to understand the images captured, extract the relevant features and then map them to their respective character representations. The Malayalam characters obtained are then translated into the English language. Once the destinations are obtained, the user is provided with the bus stops through which that particular bus travels. Also the user is made aware of the user’s destination.  
\end{flushleft}

\begin{flushleft}
\vspace{1pt}
\begin{tabular}{|c|c|}
\hline
\textbf{Presented by} \hspace{70pt} & Mini Aishwarya (SCT16CS039) \\ & Merin Cherian (SCT16CS038) \\ & Amaya Vijayan (SCT16CS011) \\
\hline
\textbf{Batch} \hspace{70pt} & R7 \\
\hline
\textbf{Guided by} \hspace{70pt} & Mr.Rejimoan R. \\
 & (Assistant Professor, Dept. of CSE, SCTCE) \\
\hline
\end{tabular}
\end{flushleft}

\begin{flushleft}
\begin{thebibliography}{2}

\bibitem{Ai}
Pranav P. Nair ; Ajay James ; C. Saravanan, \textit{Malayalam handwritten character recognition using convolutional neural network}, International Conference on Inventive Communication and Computational Technologies (ICICCT), 2017

\end{thebibliography}
\end{flushleft}
\end{document}
